\documentclass[final,3p,times,pdflatex]{elsarticle}
%\usepackage{axodraw}
\usepackage{amsmath}
\usepackage{amssymb}
\usepackage{graphicx}
\usepackage{color} 
\definecolor{Red}{cmyk}{0,1,1,0}
\definecolor{Blue}{cmyk}{1,0.75,0,0}

\newcommand\beq{\begin{eqnarray}}
\newcommand\eeq{\end{eqnarray}}


% beginning of macros
\def\SOFTSUSY{{\tt SOFTSUSY}}
\def\code#1{{\tt #1}}

\journal{Computer Physics Communications}

\begin{document}

\begin{frontmatter}

\begin{flushright}
DAMTP-2016-??\\
\end{flushright}

\title{The Calculation of Sparticle Decays in Minimal and Next-to-Minimal Supersymmetric Standard Models: {\tt SOFTSUSY4.0}}

\author[damtp]{B.C.~Allanach}
\cortext[cor1]{Corresponding author}
\author[damtp]{T.~Cridge}
\ead{T.Cridge@damtp.cam.ac.uk}
\address[damtp]{DAMTP, CMS, University of Cambridge, Wilberforce road,
  Cambridge, CB3  0WA, United Kingdom}
\begin{abstract}
We describe a major extension of the {\tt SOFTSUSY} spectrum calculator to
include 
the calculation of the decays of branching ratios and lifetimes of sparticles
into lighter 
sparticles, covering in the next-to-minimal supersymmetric standard model as
well as the minimal supersymmetric standard model.
This document
provides an overview of the program and acts as a manual for the
new version of {\tt SOFTSUSY}, which includes the calculation of sparticle
decays. 
\end{abstract}

\begin{keyword}
MSSM, NMSSM, branching ratio, lifetime
\PACS 12.60.Jv
\PACS 14.80.Ly
\end{keyword}
\end{frontmatter}

\section{Program Summary}
\noindent{\em Program title:} \SOFTSUSY{} \\
{\em Program obtainable   from:} {\tt http://softsusy.hepforge.org/} \\
{\em Distribution format:}\/ tar.gz \\
{\em Programming language:} {\tt C++}, {\tt fortran} \\
{\em Computer:}\/ Personal computer. \\
{\em Operating system:}\/ Tested on Linux 4.4.0-36-generic, Linux 3.13.0-93-generic
\\
{\em Word size:}\/ 64 bits. \\
{\em External routines:}\/ None \\
{\em Typical running time:}\/ 0.1-1 seconds per parameter point. \\
{\em Nature of problem:}\/ Calculating supersymmetric particle decays in the
MSSM or the NMSSM\@, given the parameters and spectrum which has already been
calculated by \SOFTSUSY{}, which has already solved the renormalisation group
eqations for a two-boundary problem (i.e.\ phenomenological constraints at the
weak 
scale and a theoretical boundary condition on soft supersymmetry breaking
terms at a high scale). \\
{\em Solution method:}\/ Analytic expressions for tree-level 2 body decays and
numerical integration for 3 body and loop-level decays.\\
{\em Restrictions:} \SOFTSUSY~will provide a solution only in the
perturbative regime and it
assumes that all couplings of the model are real
(i.e.\ $CP-$conserving). If the parameter point under investigation is
non-physical for some reason (for example because the electroweak potential
does not have an acceptable minimum), \SOFTSUSY{} returns an error message.
The higher order corrections included are for the 
MSSM ($R-$parity conserving or violating) or the real $R-$parity conserving
NMSSM only. \\
{\em CPC Classification:}\/ 11.1 and 11.6. \\
{\em Does the new version supersede the previous version?:}\/ Yes. \\
{\em Reasons for the new version:}\/ Significantly extended functionality. The
decay rates and branching ratios of sparticles are particularly useful for
collider searches. Also, we believe that there is not enough competition and
consequent checks and balances for decays in the next-to-minimal
supersymmetric standard model.\\
{\em Summary of revisions:}\/
Addition of the calculation of sparticle decays. 
All 2- and important 3-body tree-level
decays, including phenomenologically important loop-level decays

\section{Introduction}

\cite{Allanach:2001kg,Allanach:2013kza,Allanach:2009bv,Allanach:2011de,Allanach:2014nba,Allanach:2016rxd}

\section*{Acknowledgments}
This work has been partially supported by STFC HEP Consolidated grant 
ST/L000385/1. We thank the Cambridge SUSY working group for helpful
discussions. 

\appendix

\section{Running \SOFTSUSY~to Calculate Sparticle Decays As Well As the Spectrum}  
\label{sec:run}

\SOFTSUSY~produces an executable called \code{softpoint.x}. 

\section{Expressions Used For Partial Widths}



\bibliography{decays}
\bibliographystyle{elsarticle-num}
\end{document}
