\documentclass[final,3p,times,pdflatex]{elsarticle}
%\usepackage{axodraw}
\usepackage{amsmath}
\usepackage{amssymb}
\usepackage{graphicx}
\usepackage{color} 
\definecolor{Red}{cmyk}{0,1,1,0}
\definecolor{Blue}{cmyk}{1,0.75,0,0}

\newcommand\beq{\begin{eqnarray}}
\newcommand\eeq{\end{eqnarray}}


% beginning of macros
\def\SOFTSUSY{{\tt SOFTSUSY}}
\def\code#1{{\tt #1}}

\journal{Computer Physics Communications}

\begin{document}

\begin{frontmatter}

\begin{flushright}
DAMTP-2017-??\\
\end{flushright}

\title{The Calculation of Sparticle Decays in Minimal and Next-to-Minimal Supersymmetric Standard Models: {\tt SOFTSUSY4.0}}

\author[damtp]{B.C.~Allanach}
\cortext[cor1]{Corresponding author}
\author[damtp]{T.~Cridge}
\ead{T.Cridge@damtp.cam.ac.uk}
\address[damtp]{DAMTP, CMS, University of Cambridge, Wilberforce road,
  Cambridge, CB3  0WA, United Kingdom}
\begin{abstract}
We describe a major extension of the {\tt SOFTSUSY} spectrum calculator to
include 
the calculation of the decays of branching ratios and lifetimes of sparticles
into lighter 
sparticles, covering  the next-to-minimal supersymmetric standard model
(NMSSM) as 
well as the minimal supersymmetric standard model (MSSM).
This document
acts as a manual for the
new version of {\tt SOFTSUSY}, which includes the calculation of sparticle
decays. We collect explicit expressions for the various partial widths in the
appendix. 
\end{abstract}

\begin{keyword}
MSSM, NMSSM, branching ratio, lifetime
\PACS 12.60.Jv
\PACS 14.80.Ly
\end{keyword}
\end{frontmatter}

\section{Program Summary}
\noindent{\em Program title:} \SOFTSUSY{} \\
{\em Program obtainable   from:} {\tt http://softsusy.hepforge.org/} \\
{\em Distribution format:}\/ tar.gz \\
{\em Programming language:} {\tt C++}, {\tt fortran} \\
{\em Computer:}\/ Personal computer. \\
{\em Operating system:}\/ Tested on Linux 4.4.0-36-generic, Linux 3.13.0-93-generic
\\
{\em Word size:}\/ 64 bits. \\
{\em External routines:}\/ None \\
{\em Typical running time:}\/ 0.1-1 seconds per parameter point. \\
{\em Nature of problem:}\/ Calculating supersymmetric particle partial decay
widths in the 
MSSM or the NMSSM\@, given the parameters and spectrum which has already been
calculated by \SOFTSUSY{}. \\
{\em Solution method:}\/ Analytic expressions for tree-level 2 body decays and
one-dimensional numerical integration for 3 body and loop-level decays.\\
{\em Restrictions:} Decays are only calculated in the real $R-$parity conserving
MSSM  or the real $R-$parity conserving
NMSSM only. \\
{\em CPC Classification:}\/ 11.1 and 11.6. \\
{\em Does the new version supersede the previous version?:}\/ Yes. \\
{\em Reasons for the new version:}\/ Significantly extended functionality. The
decay rates and branching ratios of sparticles are particularly useful for
collider searches. Decays calculated in the NMSSM will be a particularly
useful check of the other programs in the literature.\\
{\em Summary of revisions:}\/
Addition of the calculation of sparticle decays. 
All 2-body and important 3-body tree-level
decays, including phenomenologically important loop-level decays (notably,
higgs decays to $gg$ and $\gamma \gamma$). Loop corrections are added to higgs
decays to $q \bar q$ for quarks $q$ of any flavour.

\section{Introduction}

\cite{Allanach:2001kg,Allanach:2013kza,Allanach:2009bv,Allanach:2011de,Allanach:2014nba,Allanach:2016rxd}

\section{Conventions and Method}
{\em Tom: Please set up conventions for use later in the appendix. You can use
any specified in the other SOFTSUSY manuals, provided you explicitly refer to
them, saying that's what you're using.}

For two body decay modes, partial widths are explicitly coded in order to
provide fast evaluation. For three body decay modes, the phase space integral
has been reduced to a one-dimensional integral analytically, which is then
performed using adaptive Gaussian numerical integration~\cite{numRec}.

\section*{Acknowledgments}
This work has been partially supported by STFC HEP Consolidated grant 
ST/L000385/1. We thank the Cambridge SUSY working group for helpful
discussions. 

\appendix

\section{Running \SOFTSUSY~to Calculate Sparticle Decays}  
\label{sec:run}

\SOFTSUSY~produces an executable called \code{softpoint.x}. It can be run by
the command
\begin{verbatim}
./softpoint.x leshouches < inOutFiles/lesHouchesInput
\end{verbatim}
where the file \code{inOutFiles/lesHouchesInput}~contains an ASCII file
for input
prepared in SUSY Les Houches Accord (SLHA)~\cite{Skands:2003cj} or
SLHA2~\cite{Allanach:2008qq} format. 
A \code{SOFTSUSY}-specific \code{Block}~of the SLHA input file is provided in
the case that decays are required:
\begin{verbatim}
Block SOFTSUSY               # Optional SOFTSUSY-specific parameters
    0   1.000000000e+00      # Calculate decays in output (only for RPC (N)MSSM)
\end{verbatim}
If this block is absent, or if the numerical value is instead
\code{0.000000000e+00}, then decays will not be calculated. 
Other input options are available for changing the behaviour of the program as
regards sparticle decays:
\begin{verbatim}
Block SOFTSUSY               # Optional SOFTSUSY-specific parameters
   24   1.000000000e-06      # If decay BR is below this number, don't output
   25   1.000000000e+00	     # If set to 0, don't calculate 3-body decays (1=default)
\end{verbatim}
Thus, parameter \code{24}~under this block sets the smallest decay branching
ratio that will be output (the default is, as listed, $10^{-6}$) whereas 
parameter \code{25}~can be used to instruct \code{SOFTSUSY}~not to calculate
the 3-body modes, which are more computationally intensive, requiring
numerical integration. 

If command-line input is required (see
Refs.~\cite{Allanach:2001kg,Allanach:2013kza} 
for a full description), the user can use the argument 
\code{--decays}~to tell \SOFTSUSY~to perform the decay calculation. 
The command line argument \code{--minBR=<value>}~tells SOFTSUSY which minimum
branching ratio to print out in each decay table (where \code{<value>} is
replaced by a numerical value between \code{0}~and \code{1}), whereas the
command line argument \code{--dontCalculateThreeBody}~tells SOFTSUSY {\em
  not}\/ to calculate the 3-body decays of sparticles in order to save time. 

\section{Sample Output}
The output comes in the standard SLHA/SLHA2
format~\cite{Skands:2003cj,Allanach:2008qq}. A sample of the part relevant for
decays is shown here:
{\small
\begin{verbatim}
#     PDG         Width             
DECAY 1000021     1.93499813e+01   # gluino decays
#     PW                BR                NDA      PDG1        PDG2        PDG3       
      3.02052144e-01    1.56099450e-02      2     1          -1000001     0         # ~g -> d ~d_L*
      3.02052144e-01    1.56099450e-02      2    -1           1000001     0         # ~g -> db ~d_L
      6.40860115e-01    3.31194178e-02      2     1          -2000001     0         # ~g -> d ~d_R*
      6.40860115e-01    3.31194178e-02      2    -1           2000001     0         # ~g -> db ~d_R
      3.20263354e-01    1.65510937e-02      2     2          -1000002     0         # ~g -> u ~u_L*
      3.20263354e-01    1.65510937e-02      2    -2           1000002     0         # ~g -> ub ~u_L
      6.11876735e-01    3.16215672e-02      2     2          -2000002     0         # ~g -> u ~u_R*
      6.11876735e-01    3.16215672e-02      2    -2           2000002     0         # ~g -> ub ~u_R
      3.02068273e-01    1.56107786e-02      2     3          -1000003     0         # ~g -> s ~s_L*
      3.02068273e-01    1.56107786e-02      2    -3           1000003     0         # ~g -> sb ~s_L
      6.40883736e-01    3.31206386e-02      2     3          -2000003     0         # ~g -> s ~s_R*
      6.40883736e-01    3.31206386e-02      2    -3           2000003     0         # ~g -> sb ~s_R
      3.20253636e-01    1.65505915e-02      2     4          -1000004     0         # ~g -> c ~c_L*
      3.20253636e-01    1.65505915e-02      2    -4           1000004     0         # ~g -> cb ~c_L
      6.11875992e-01    3.16215288e-02      2     4          -2000004     0         # ~g -> c ~c_R*
      6.11875992e-01    3.16215288e-02      2    -4           2000004     0         # ~g -> cb ~c_R
      1.04760821e+00    5.41400117e-02      2     5          -1000005     0         # ~g -> b ~b_1*
      1.04760821e+00    5.41400117e-02      2    -5           1000005     0         # ~g -> bb ~b_1
      6.79062885e-01    3.50937231e-02      2     5          -2000005     0         # ~g -> b ~b_2*
      6.79062885e-01    3.50937231e-02      2    -5           2000005     0         # ~g -> bb ~b_2
      4.19818556e+00    2.16960704e-01      2     6          -1000006     0         # ~g -> t ~t_1*
      4.19818556e+00    2.16960704e-01      2    -6           1000006     0         # ~g -> tb ~t_1
      [ ... ]
\end{verbatim}}
In concordance with SLHA conventions, the widths/partial widths
(\code{PW}) are output 
in units 
of GeV. \code{NDA}~lists the number of daughter particles and
\code{PDGi}~lists the
Particle Data Group~(PDG) code of daughter \code{i} (Section 43 of
Ref.~\cite{Olive:2016xmw}). 
The comment at the end of each line after \code{\#}~lists the decay
mode for easy perusal by the eye.
\section{Explicit Expressions Used For Partial Widths}



\bibliography{decays}
\bibliographystyle{elsarticle-num}
\end{document}
