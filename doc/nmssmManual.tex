\documentclass[final,3p,times,pdflatex]{elsarticle}
\usepackage{axodraw}

\bibstyle{elsarticle-num}

% beginning of macros
\def\SOFTSUSY{{\tt SOFTSUSY}}
\def\code#1{\small{\tt #1}\normalsize}
% end of macros

\journal{Computer Physics Communications}

\begin{document}

\begin{frontmatter}

\title{Next-to-Minimal SOFTSUSY}

\author{B.C.~Allanach}
\address{DAMTP, CMS, University of Cambridge, Wilberforce road, Cambridge, CB3
  0WA, United Kingdom}

\author{P.~Athron}
\author{A.~Williams}
\author{L.~Tunstall}
\address{Department of Physics, North Terrace, University of Adelaide,
  Adelaide SA 5005, Australia}

\author{A.~Voigt}
\address{Dipl.-Phys. Alexander Voigt,
Institut für Kern- und Teilchenphysik,
Helmholtzstraße 10, 
D-01069 Dresden,
Germany}
\begin{abstract}
  Here, we describe an extension to the
  {\tt SOFTSUSY}~program to calculate the sparticle spectrum in the
  {\em next-to-minimal} supersymmetric standard model, where a chiral
  superfield that is a singlet of the Standard Model gauge group is added to
  the minimal supersymmetric standard model fields. 
  The user provides a theoretical boundary condition upon the couplings and
  masses of the singlet.
  Successful radiative electroweak symmetry breaking,
  electroweak and CKM matrix data are used
  as weak-scale boundary conditions. 
  The renormalisation group equations are solved
  numerically between the weak scale and a high energy scale using a nested
  iterative algorithm. 
  This paper serves as a manual to the
  next-to-minimal mode of the program, detailing the approximations and
  conventions used. 
\end{abstract}

\begin{keyword}
sparticle, 
NMSSM, Higgs
\PACS 12.60.Jv
\PACS 14.80.Ly
\end{keyword}
\end{frontmatter}

\section{Program Summary}
\noindent{\em Program title:} \SOFTSUSY{}\\
{\em Program obtainable
  from:} {\tt http://projects.hepforge.org/softsusy/}\\
{\em Distribution format:}\/ tar.gz\\
{\em Programming language:} {\tt C++}, {\tt fortran}\\
{\em Computer:}\/ Personal computer\\
{\em Operating system:}\/ Tested on Linux 4.x\\
{\em Word size:}\/ 32 bits\\
{\em External routines:}\/ None\\
{\em Typical running time:}\/ a few seconds per parameter point.\\
{\em Nature of problem:}\/ Calculating supersymmetric particle spectrum and
mixing parameters in the next-to-minimal minimal supersymmetric standard
model. The solution to the renormalisation group equations must be consistent
with boundary conditions on supersymmetry breaking parameters, as
well as a weak-scale boundary condition on gauge 
couplings, Yukawa couplings and the Higgs potential parameters.\\
{\em Solution method:}\/ Nested iterative algorithm and numerical minimisation
of the Higgs potential. \\
{\em Restrictions:} {\SOFTSUSY} will provide a solution only in the
perturbative r\'{e}gime and it
assumes that all couplings of the model are real
(i.e.\ $CP-$conserving). The iterative \SOFTSUSY~algorithm will not 
converge if parameters are too close to a boundary of successful electroweak
symmetry breaking, but a warning flag will alert the user to this fact.

\newpage

\section{Introduction}

Recently, an apparently Standard Model Higgs boson was discovered in the CMS
and ATLAS experiments at over the 5$-\sigma$ level~\cite{}. 

\section{NMSSM Parameters \label{sec:notation}}

In this section, we introduce the NMSSM parameters
in the \SOFTSUSY~conventions. The translations to the actual variable
names that are being used in the program code are shown explicitly in
appendix~\ref{sec:objects}. 

\subsection{Supersymmetric parameters \label{susypars}}
The chiral superfield particle content of the NMSSM has the 
following $SU(3)_c\times SU(2)_L\times U(1)_Y$ quantum numbers:
\begin{eqnarray}
L:&(1,2,-\frac{1}{2}),\quad {\bar E}:&(1,1,1),\qquad\, Q:\,(3,2,\frac{1}{6}),\quad
{\bar U}:\,(\bar 3,1,-\frac{2}{3}),\nonumber\\ {\bar D}:&(\bar 3,1,\frac{1}{3}),\quad
H_1:&(1,2,-\frac{1}{2}),\quad  H_2:\,(1,2,\frac{1}{2}),\quad N:\,(1,1,0)
\label{fields}
\end{eqnarray}
$N$ is the Higgs singlet that is particular to the NMSSM.
$L$, $Q$, $H_1$, and $H_2$ are the left handed doublet lepton and
quark superfields and the two Higgs doublets. $\bar E$, $\bar U$, and
$\bar D$ are the lepton, up-type quark and down-type quark
right-handed superfield singlets, respectively. 
Note that the lepton
doublet superfields $L^a_i$ and the Higgs doublet superfield coupling
to the down-type quarks, $H_1$, have the same SM gauge 
quantum numbers. 
Here, we denote an $SU(3)$ colour index of the fundamental
representation by  $\{x,y,z\} \in \{1,2,3 \}$. The $SU(2)_L$ fundamental
representation indices are denoted by $\{a,b,c\} \in \{1,2\}$ and the generation
indices by $\{i,j,k\} \in \{1,2,3\}$. 
$\epsilon_{xyz}=\epsilon^{xyz}$ and  $\epsilon_{ab}=\epsilon^{ab}$ are totally
antisymmetric tensors, with $\epsilon_{123}=1$ and $\epsilon_{12}=1$,
respectively.  Currently, 
only real couplings in the superpotential and Lagrangian are included. 

\subsection{Next-to-Minimal SUSY breaking parameters \label{sec:susybreak}}


\subsection{Tree-level masses \label{sec:tree}}

\section{Calculation Algorithm \label{sec:calculation}}


\begin{figure}
\begin{center}
\begin{picture}(323,245)
\put(10,0){\makebox(280,10)[c]{\fbox{7.\ Calculate Higgs and
      sparticle pole masses. Run to $M_Z$.}}}
\put(10,40){\makebox(280,10)[c]{\fbox{6.\ Run to $M_Z$.}}}
\put(150,76.5){\vector(0,-1){23}}
\put(10,80){\makebox(280,10)[c]{\fbox{5.\ Run to $M_X$. Apply soft breaking
and NMSSM SUSY boundary conditions.}}}
\put(150,116.5){\vector(0,-1){23}}
\put(10,120){\makebox(280,10)[c]{\fbox{4.\ EWSB, iterative solution of $\mu$ and sneutrino VEVs.}}}
\put(150,156){\vector(0,-1){23}}
\put(10,160){\makebox(280,10)[c]{\fbox{3.\ Run to $M_S$.}}}
\put(30,170){convergence}
\DashLine(115,165)(0,165){5}
\DashLine(0,165)(0,5){5}
\DashLine(0,5)(30,5){5}
\put(30,5){\vector(1,0){2}}
\put(150,197){\vector(0,-1){24}}
\put(150,239){\vector(0,-1){26}}
\put(60,245){\fbox{1.\ SUSY radiative corrections to
$g_i(M_Z)$}}
\put(10,200){\makebox(280,10)[c]{\fbox{2.\ $Y_E$, $Y_D$ include $v_k$
      contributions. Iterative solution of $Y_E$.}}} 
\put(182,45){\line(1,0){161}}
\put(343,45){\line(0,1){200}}
\put(343,245){\vector(-1,0){115}}
\end{picture}
\end{center}
\caption{Iterative algorithm used to calculate the NMSSM spectrum. 
The initial step is the
uppermost one. $M_S$ is the scale at which the EWSB
conditions 
are imposed, as discussed in the text. $M_X$ is the scale at which the high
energy SUSY breaking boundary conditions are imposed.
\label{fig:algorithm}}
\end{figure}


\subsection{Running of NMSSM couplings~\label{running}}
The program 
can be made to run faster by switching off the two-loop renormalisation of 
the scalar masses and tri-linear scalar couplings. All
$\beta$ functions are real and include inter-generational quark mixing
effects. 

\subsection{Electroweak symmetry breaking \label{ewsb}}


\subsection{MMSSM spectrum \label{spec}}



\section*{Acknowledgments}
This work has been partially supported by 
STFC\@. 


\appendix

\section{Running \SOFTSUSY}
\label{sec:run}

\SOFTSUSY~produces an executable called \code{softpoint.x}. For the calculation
of the spectrum of single points in parameter space, we recommend the
SUSY Les Houches Accord 2 (SLHA2)~\cite{Allanach:2008qq}  input/output
option. The user must provide a file (\textit{e.g.} the example file included
in the \SOFTSUSY~distribution
\code{rpvHouchesInput}), that specifies the model dependent input
parameters. The program may then be run with
\small
\begin{verbatim}
 ./softpoint.x leshouches < nmssmHouchesInput
\end{verbatim}
\normalsize
For the SLHA2 input option, 
the output will also be given in 
SLHA2 format. Such output can be used for
input into other programs which subscribe to the accord, such as
\code{PYTHIA}~\cite{Sjostrand:2007gs} (for
simulating sparticle production and decays at colliders), for example. For
further details on the necessary format of 
the input file, see ref.~\cite{Allanach:2008qq}. It supports 
the setting of all other SLHA2 input blocks associated with non-complex
couplings. 

$M_{GUT}$ is used (defined to be the $\overline{DR}$ scale $Q$ at which $g_1(Q) =
g_2(Q)$). The default output is in SLHA2 format, the conventions of which are
explained in Ref.~\cite{Allanach:2008qq}. 


\section{Object Structure\label{sec:objects}}

We now go on to sketch the objects and their relationship to each other. This
is necessary information for any generalisation. Only
methods and data which are deemed of possible importance for prospective users
are 
mentioned here, but there are many others within the program itself.

\subsection{NmssmSoftsusy}
\label{nmssmsoftsusy}


\subsection{General object structure}

From an RGE point of view, data in a particular quantum field theory 
consist of a set of parameters defined at some
renormalisation scale $Q$. 
A set of $\beta$ functions describes the
evolution of the parameters and masses to a different scale
$Q'$. This concept is embodied in an {\em abstract} \code{RGE}
object, which contains the methods required to run objects of derived
classes to different renormalisation scales (their beta functions). The other
objects 
displayed in figure~\ref{fig:objstruc} are particular instances of
\code{RGE}, and therefore inherit from it. \code{QedQcd}, \code{MssmSusy},
\code{SoftParsMssm} and \code{MssmSoftsusy}~objects encode the $R_p$ part of the
MSSM and its SM input data~\cite{Allanach:2001kg}.
\code{RpvSusyPars} contains all of the supersymmetric  couplings contained
within  Eq.~(\ref{superpot1}). 
\code{RpvSoftPars}~contains the  soft supersymmetry breaking parameters
listed in 
Eqs.~(\ref{softhabertril}) and~(\ref{softhaberbil}). \code{RpvSoftsusy} is
the  $R-$parity violating
generalisation of the \code{MssmSoftsusy}~class. 
Code in
the \code{MssmSoftsusy} class organises and performs the main part of
the calculation, using polymorphism to detect the correct $\beta$ functions to
use (in this case, the $-$NMSSM $\beta$ functions). 
All of the \code{Nmssm}~objects contain default constructors
and destructors, as well as overloaded \code{>>},\code{<<}~operators for input
and output. 
There is always an implicit dependence of running RGE quantities on the
current renormalisation scale $Q$. Thus, if a method is called that returns
one of the object's couplings or masses, that object will return it at the
current scale $Q$ of the object. 
In the following, we provide basic information on the 
classes associated with , so that users may program using the class
structure of \SOFTSUSY\@. More detailed and
technical documentation on the program should be obtained from the
\SOFTSUSY~website. 


\subsection{\code{NmssmSusyPars}~class}

\begin{thebibliography}{10}
\end{thebibliography}

\end{document}
